\documentclass{article}

% set up telugu
\usepackage{fontspec}
\newfontfamily\telugufont{Potti Sreeramulu}[Script = Telugu,RawFeature={mode=harf}]
\usepackage{polyglossia}
\setdefaultlanguage{english}
\setotherlanguage{telugu}

%other packages
\usepackage{amsmath}
\usepackage{amssymb}
\usepackage{physics}
\usepackage{siunitx}
\usepackage[plain]{fancyref}
\usepackage{luacode}
\usepackage{titling}

% custom deepak packages
\usepackage{luatrivially}
\usepackage{subtitling}

% Extra math.random() just advances counter one to make things less repetitive.
\begin{luacode*}
	math.randomseed(3141592)
	math.random()
\end{luacode*}

% I like my vectors bold, not arrowed
\renewcommand{\vec}[1]{\mathbf{#1}}
\newcommand{\ham}{\hat{H}}
\newcommand{\msr}{\mathcal{D}}

\title{Problem 2.1.6}
\subtitle{Harmonic oscillator propagator}
\author{\begin{telugu}హృదయ్ దీపక్ మల్లుభొట్ల\end{telugu}}
% want empty date
\predate{}
\date{}
\postdate{}

% !TeX spellcheck = en_GB
\begin{document}
	\maketitle
	Show that the propagator of a harmonic oscillator has the form
	\begin{equation}
		G(x_b, t, x_a, 0) = A(t) \exp(\frac{i m \omega_0}{2 \sin(t \omega_0)} \left[ (x_b^2 + x_a^2)\cos(\omega_0 t) - 2 x_b x_a \right]).
	\end{equation}
	Use the normalization condition or the path integral to show that
	\begin{equation}
		A(t) = \left( \frac{m \omega_0}{2 \pi i\sin(t\omega_0)} \right)^{\flatfrac{1}{2}} e^{i \phi(t)}
	\end{equation}
	\section{Solution} \label{sec:solution}


\end{document}
