\documentclass{article}

% set up telugu
\usepackage{fontspec}
\newfontfamily\telugufont{Potti Sreeramulu}[Script = Telugu,RawFeature={mode=harf}]
\usepackage{polyglossia}
\setdefaultlanguage{english}
\setotherlanguage{telugu}

%other packages
\usepackage{amsmath}
\usepackage{amssymb}
\usepackage{physics}
\usepackage{siunitx}
\usepackage[plain]{fancyref}
\usepackage{luacode}
\usepackage{titling}

% custom deepak packages
\usepackage{luatrivially}
\usepackage{subtitling}

% Extra math.random() just advances counter one to make things less repetitive.
\begin{luacode*}
	math.randomseed(3141592)
	math.random()
\end{luacode*}

% I like my vectors bold, not arrowed
\renewcommand{\vec}[1]{\mathbf{#1}}
\newcommand{\ham}{\hat{H}}
\newcommand{\msr}{\mathcal{D}}

\title{Problem 2.1.3}
\subtitle{Coherent-state path integral: gory}
\author{\begin{telugu}హృదయ్ దీపక్ మల్లుభొట్ల\end{telugu}}
% want empty date
\predate{}
\date{}
\postdate{}

% !TeX spellcheck = en_GB
\begin{document}
	\maketitle
	Consider a harmonic oscillator $H = \omega \hat{a}^\dagger \hat{a}$.
	A coherent state $\ket{a}$ labeled by a complex number $a$ is an eigenstate of $\hat{a}$ : $\hat{a}\ket{a} = a \ket{a}$.
	We may define a coherent-state propagator by
	\begin{equation}
		i G(a_b, t_b, a_a, t_a) = \bra{a_b} U(t_b, t_a) \ket{a_a}.
	\end{equation}
	Using the completeness of the coherent state
	\begin{equation}
		\int \frac{\dd[2]{a}}{\pi}\dyad{a}{a} = 1
	\end{equation}
	(where $\dd[2]{a} = \dd{\Re(a)} \dd{\Im(a)}$) and the inner product
	\begin{equation}
		\bra{a}\ket{a'} = e^{-\flatfrac{\abs{a}^2}{2}} e^{-\flatfrac{\abs{a'}^2}{2}} e^{a^\ast a'}, \label{eq:innerproduct}
	\end{equation}
	show that the path integral representation of $G$ is
	\begin{equation}
		i G(a_b, t_b, a_a, t_a) = \int \pi^{-N + 1} \msr^2[a(t)] e^{i \int_{t_a}^{t_b} \dd{t} \left[i \frac12 \left(a^\dagger \dot{a} - a \dot{a}^\dagger \right) - \omega a^\dagger a \right]}, \label{eq:pathintegraltarget}
	\end{equation}
	where $\msr^2[a(t)] = \prod_{j=1}^{N-1} \dd[2]a_j$.
	Show that the Lagrangian in the coherent-state path integral
	\begin{equation}
		L = i\frac12 \left(a^\dagger \dot{a} - a \dot{a}^\dagger \right) - \omega a^\dagger a
	\end{equation}
	is simply the Lagrangian in the phase-space path integral
	\begin{equation}
		L = p\dot{x} - H
	\end{equation}
	up to a total time derivative term.

	\section{Solution} \label{sec:solution}

	\subsection{Path integral representation} \label{subsec:pathintegralrep}
	We can start by showing~\eqref{eq:pathintegraltarget}.
	In general
	\begin{align}
		i G(a_2, t_2, a_1, t_1) &= \bra{a_2} U(t_2, t_1) \ket{a_1} \\
		i G(a_2, t_2, a_1, t_1) &= \bra{a_2} e^{-i H (t_2 - t_1)} \ket{a_1} \\
		i G(a_2, t_2, a_1, t_1) &= \bra{a_2} e^{-i \omega \hat{a}^\dagger \hat{a} (t_2 - t_1)} \ket{a_1}.
	\end{align}
	\triv for a given time slice (and with $x$ representing $a$ and $t$)
	\begin{align}
		i G(x_i, x_{i-1}) &= \bra{a_i} e^{-i \omega \hat{a}^\dagger \hat{a} (t_i - t_{i-1})} \ket{a_{i-1}} \\
		&= \bra{a_i} \ket{a_{i-1}}  e^{-i \omega a_i^\ast a_{i-1} \Delta t}
	\end{align}
	Inserting~\eqref{eq:innerproduct},
	\begin{align}
		i G(x_i, x_{i-1}) &= e^{-\flatfrac{\abs{a_i}^2}{2}} e^{-\flatfrac{\abs{a_{i-1}}^2}{2}} e^{a_i^\ast a_{i-1}} e^{-i \omega a_i^\ast a_{i-1} \Delta t} \\
		&= e^{-\frac{1}{2} \left( \abs{a_i}^2 + \abs{a_{i-1}}^2 - 2 a_i^\ast a_{i-1}\right)} e^{-i \omega a_i^\ast a_{i-1} \Delta t} \\
		&= e^{i \left(\frac{i}{2 \Delta t} \left( \abs{a_i}^2 + \abs{a_{i-1}}^2 - 2 a_i^\ast a_{i-1}\right)\right) \Delta t}  e^{-i \omega a_i^\ast a_{i-1} \Delta t} \label{eq:fullexponential}
	\end{align}
	Simplifying the argument of the first exponential,
	\begin{align}
		& \frac{i}{2 \Delta t} \left( \abs{a_i}^2 + \abs{a_{i-1}}^2 - 2 a_i^\ast a_{i-1}\right)\\
		&= \frac{i}{2 \Delta t} \left( a_i^\ast a_i + a_{i-1}^\ast a_{i-1} - 2 a_i^\ast a_{i-1}\right)\\
		&= \frac{i}{2 \Delta t} \left( a_i^\ast a_i - a_i^\ast a_{i-1} - a_i^\ast a_{i-1}+ a_{i-1}^\ast a_{i-1}\right)\\
		&= \frac{i}{2 \Delta t} \left( a_i^\ast \left(a_i -  a_{i-1}\right) - \left(a_i^\ast - a_{i-1}^\ast \right)a_{i-1}\right)\\
		&= \frac{i}{2} \left( a^\ast \frac{\Delta a}{\Delta t} - \frac{\Delta a^\ast}{\Delta t} a \right) \\
		&= \frac{i}{2} \left( a^\ast \dot{a} -  \dot{a}^\ast  a \right)
	\end{align}
	where in the penultimate expression we've dropped the subscripts to anticipate the $\Delta t \rightarrow 0$ limit.
	Plugging this back into~\eqref{eq:fullexponential}, while simultaneously taking the same limit in the second exponential gives us
	\begin{align}
		i G(x_i, x_{i-1}) &= e^{i \left( \frac{i}{2} \left( a^\ast \dot{a} -  \dot{a}^\ast  a \right) \right) \Delta t}  e^{-i \omega a^\ast a \Delta t} \\
		&= e^{i \left( \frac{i}{2} \left( a^\ast \dot{a} -  \dot{a}^\ast  a \right) - \omega a^\ast a \right) \Delta t} \label{eq:oneslice}
	\end{align}

	Now that we know what's up with a single time slice, let's relate that to our original case by taking time and violently slashing it to ribbons.
	\triv for $N = 2$ slices,
	\begin{align}
		i G(a_b, t_b, a_a, t_a) &= \bra{a_b} U(t_b, t_a) \ket{a_a} \\
		i G(a_b, t_b, a_a, t_a) &= \bra{a_b} U(t_b, t_1) U(t_1, t_a) \ket{a_a}
	\end{align}
	\triv we can insert the identity here:
	\begin{align}
		i G(a_b, t_b, a_a, t_a) &= \int \frac{\dd[2]{a_1}}{\pi} \bra{a_b} U(t_b, t_1) \ket{a_1} \bra{a_1} U(t_1, t_a) \ket{a_a} \\
		i G(a_b, t_b, a_a, t_a) &= \int \frac{\dd[2]{a_1}}{\pi} i G(a_b, t_b, a_1, t_1)  i G(a_1, t_1, a_a, t_a)
	\end{align}
	As we divide this into $N$ time slices, we'll end up with $N - 1$ integrals to perform, and thus, as our number of time slices get large enough that~\eqref{eq:oneslice} is valid,
	\begin{align}
		i G(a_b, t_b, a_a, t_a) &= \int \prod_{j = 1}^{N - 1} \left( \frac{\dd[2]{a_j}}{\pi}\right)  \prod_{i=1}^{N} i G(a_i, t_i, a_{i-1}, t_{i-1}) \\
		i G(a_b, t_b, a_a, t_a) &= \frac{1}{\pi^{N - 1}} \int \msr^2[a(t)] \prod_{i=1}^{N} i G(a_i, t_i, a_{i-1}, t_{i-1}) \\
		i G(a_b, t_b, a_a, t_a) &= \frac{1}{\pi^{N - 1}} \int \msr^2[a(t)] e^{i \sum_t \left( \frac{i}{2} \left( a^\ast \dot{a} -  \dot{a}^\ast  a \right) - \omega a^\ast a \right) \Delta t} \\
		i G(a_b, t_b, a_a, t_a) &= \frac{1}{\pi^{N - 1}} \int \msr^2[a(t)] e^{i \int_{t_a}^{t_b} \dd{t} \frac{i}{2} \left( a^\ast \dot{a} -  \dot{a}^\ast  a \right) - \omega a^\ast a}
	\end{align}
	\thrf we have shown~\eqref{eq:pathintegraltarget}.
	\subsection{Lagrangian is phase-space} \label{subsec:lagrangianphasespace}
	Now, we can show that the Lagrangian
	\begin{equation}
		L = i\frac12 \left(a^\dagger \dot{a} - a \dot{a}^\dagger \right) - \omega a^\dagger a
	\end{equation}
	is equivalent to the phase space Lagrangian, up to isomophic dynamic structure.
	We essentially need to show that
	\begin{equation}
		i\frac12 \left(a^\dagger \dot{a} - a \dot{a}^\dagger \right) = p \dot{x}, \label{eq:lagrangianconnection}
	\end{equation}
	up to a total time derivative.
	We know that
	\begin{align}
		\hat{a} = \sqrt{\frac{m \omega}{2}} \left(\hat{x} + \frac{i}{m \omega} \hat{p} \right) \\
		\hat{a}^\dagger = \sqrt{\frac{m \omega}{2}} \left(\hat{x} - \frac{i}{m \omega} \hat{p} \right)
	\end{align}
	Writing $k = \frac{i}{m \omega} p$ and inserting these into~\eqref{eq:lagrangianconnection} gives us
	\begin{align}
		i\frac12 \left(a^\dagger \dot{a} - a \dot{a}^\dagger \right) &= \frac12 \frac{m \omega}{2} \left( \left( x - k \right)\left(\dot{x} + \dot{k} \right) - \left(x + k \right) \left(\dot{x} - \dot{k} \right)\right) \\
		&= i  \frac{m \omega}{2} \left( \dot{k} x - \dot{x} k \right) \\
		&= i \frac12 \frac{m \omega}{2} \left( \dot{k} x + \dot{x} k - 2 \dot{x} k \right) \\
		&= i \frac12 \frac{m \omega}{2} \left( \dv{t}(xk) - 2 \dot{x} k \right)
	\end{align}
	We can drop the first term here, as a total time derivative in a Lagrangian.
	\begin{align}
		i\frac12 \left(a^\dagger \dot{a} - a \dot{a}^\dagger \right) &= i \frac12 \frac{m \omega}{2} \left( \dv{t}(xk) - 2 \dot{x} k \right) \\
		&= i m \omega \left( - \dot{x} k \right) \\
		&= i m \omega \left( - \dot{x} \frac{i}{m \omega} p \right) \\
		&= p \dot{x}
	\end{align}
	So we done.

\end{document}
