\documentclass{article}

% set up telugu
\usepackage{fontspec}
\newfontfamily\telugufont{Potti Sreeramulu}[Script = Telugu,RawFeature={mode=harf}]
\usepackage{polyglossia}
\setdefaultlanguage{english}
\setotherlanguage{telugu}

%other packages
\usepackage{amsmath}
\usepackage{amssymb}
\usepackage{physics}
\usepackage{siunitx}
\usepackage{todonotes}
\usepackage[plain]{fancyref}
\usepackage{luacode}

% custom deepak packages
\usepackage{trivially}
\usepackage{subtitling}
\begin{luacode*}
	math.randomseed(31415926)
\end{luacode*}

\title{Problem 2.1.1}
\subtitle{On the pole structure of propagators: gory version}
\author{\begin{telugu}హృదయ్ దీపక్ మల్లుభొట్ల\end{telugu}}

% !TeX spellcheck = en_GB
\begin{document}
	\maketitle
	Show that, in the frequency space, we have
	\begin{equation}
		G(x_b, x_a, \omega) = \sum_n \frac{\psi_n(x_b) \psi^\dagger(x_a)}{\omega - \epsilon_n} \label{eq:target1}
	\end{equation}
	where $\psi_n$ are the energy eigenfunctions.
	The propagator for a harmonic oscillator has the form
	\begin{equation}
		G(x_b, t, x_a, 0) = (-i) \left( \frac{m \omega_0}{2 \pi i \sin(t \omega_0)} \right)^{\flatfrac12} \exp(\frac{i m \omega_0}{2 \pi \sin(t\omega_0)}\left[(x_b^2 + x_a^2)\cos(t \omega_0) - 2 x_b x_a \right]) \label{eq:harmonicoscprop}
	\end{equation}
	Study and explain the pole structure of $G(0, 0, \omega)$ for the harmonic oscillator.
	(Hint: Try to expand $G(0, t, 0, 0,)$ in the form $\sum C_n e^{-i t \epsilon_n}$.)

	\section{Solution} \label{sec:solution}
	\subsection{Pole structure form of propagator} \label{subsec:sol1}
	We want to begin by showing~\eqref{eq:target1}.
	The coordinate Green function is defined via
	\begin{equation}
		i G(x_b, t_b, x_a, t_a) = \bra{x_b} U(t_b, t_a) \ket{x_a}
	\end{equation}
	We can insert the identity twice:
	\begin{align}
		G(x_b, t_b, x_a, t_a) &= -i \bra{x_b} U(t_b, t_a) \ket{x_a} \\
		G(x_b, t_b, x_a, t_a) &= -i \sum_{n_b, n_a} \bra{x_b} \ket{n_b}\bra{n_b} U(t_b, t_a) \ket{n_a}\bra{n_a}  \ket{x_a}
	\end{align}
	where $\ket{n_i}$ is the $i-$th energy eigenstate.
	\triv
	\begin{equation}
		\bra{n_b} U(t_b, t_a) \ket{n_a} = e^{-i \epsilon_{n_b} (t_b - t_a)} \delta_{n_a, n_b},
	\end{equation}
	\thrf
	\begin{align}
		G(x_b, t_b, x_a, t_a) &= -i \sum_{n_b, n_a} \bra{x_b} \ket{n_b}\bra{n_b} U(t_b, t_a) \ket{n_a}\bra{n_a}  \ket{x_a} \\
		&= -i \sum_{n_b, n_a} \bra{x_b} \ket{n_b}e^{-i \epsilon_{n_b} (t_b - t_a)} \delta_{n_a, n_b}\bra{n_a}  \ket{x_a} \\
		&= -i \sum_{n_b} \bra{x_b} \ket{n_b}e^{-i \epsilon_{n_b} (t_b - t_a)} \bra{n_b}  \ket{x_a} \\
		&= -i \sum_{n} \bra{x_b} \ket{n}e^{-i \epsilon_{n} (t_b - t_a)} \bra{n}  \ket{x_a} \\
		&= -i \sum_{n} \psi_n(x_b) e^{-i \epsilon_{n} (t_b - t_a)} \psi_n^\dagger(x_a) \\
		&= -i \sum_{n} \psi_n(x_b) \psi_n^\dagger(x_a)  e^{-i \epsilon_{n} (t_b - t_a)}
	\end{align}
	In frequency space,
	\begin{align}
		G(x_b, x_a, \omega) = \int_0^\infty \dd{t} G(x_b, t_a + t, x_a, t_a) e^{i t \omega - \delta t},
	\end{align}
	where $\delta$ is some small positive constant we'll send to zero later.
	Plugging this in,
	\begin{align}
		G(x_b, x_a, \omega) &= \int_0^\infty \dd{t} \left( -i \sum_{n} \psi_n(x_b) \psi_n^\dagger(x_a)  e^{-i \epsilon_{n} t} \right) e^{i t \omega - \delta t} \\
		G(x_b, x_a, \omega) &= -i \sum_{n} \psi_n(x_b) \psi_n^\dagger(x_a)  \int_0^\infty \dd{t} e^{-i \epsilon_{n} t} e^{i t \omega - \delta t}
	\end{align}
	\triv
	\begin{align}
		G(x_b, x_a, \omega) &= -i \sum_{n} \psi_n(x_b) \psi_n^\dagger(x_a)  \int_0^\infty \dd{t} e^{-i \epsilon_{n} t } e^{i t \omega - \delta t} \\
		G(x_b, x_a, \omega) &= -i \sum_{n} \psi_n(x_b) \psi_n^\dagger(x_a)  \int_0^\infty \dd{t} e^{t (-\delta + i(\omega - \epsilon_n))} \\
		G(x_b, x_a, \omega) &= -i \sum_{n} \psi_n(x_b) \psi_n^\dagger(x_a)  \frac{1}{(-\delta + i(\omega - \epsilon_n))} \left. e^{-\delta t} \right|_{t =0}^{\infty} \\
		G(x_b, x_a, \omega) &= -i \sum_{n} \psi_n(x_b) \psi_n^\dagger(x_a)  \frac{-1}{(-\delta + i(\omega - \epsilon_n))} \\
		G(x_b, x_a, \omega) &= \sum_{n} \psi_n(x_b) \psi_n^\dagger(x_a)  \frac{i}{(-\delta + i(\omega - \epsilon_n))} \\
		G(x_b, x_a, \omega) &= \sum_{n} \frac{\psi_n(x_b) \psi_n^\dagger(x_a)}{\omega - \epsilon_n + i \delta}
	\end{align}
	\thrf after we send $\delta \rightarrow 0$, we have shown~\eqref{eq:target1}, as desired.

	\subsection{Pole structure of harmonic oscillator propagator} \label{subsec:sol2}
	We can start with~\eqref{eq:harmonicoscprop}, and set $x_b = x_a = 0$.
	\triv this gives us
	\begin{align}
		G(x_b = 0, t, x_a = 0, 0) &= (-i) \left( \frac{m \omega_0}{2 \pi i \sin(t \omega_0)} \right)^{\flatfrac12} \exp(\frac{i m \omega_0}{2 \pi \sin(t\omega_0)}\left[0\right]) \\
		&= (-i) \left( \frac{m \omega_0}{2 \pi i \sin(t \omega_0)} \right)^{\flatfrac12} \\
		iG(0, t, 0, 0) &= \sqrt{\frac{m \omega_0}{\pi}} \frac{1}{\left(2  i \sin(t \omega_0) \right)^{\flatfrac12}} \\
		iG(0, t, 0, 0) &= \sqrt{\frac{m \omega_0}{\pi}} \frac{1}{\left(2  i \sin(t \omega_0) \right)^{\flatfrac12}} \\
		iG(0, t, 0, 0) &= \sqrt{\frac{m \omega_0}{\pi}} \frac{1}{\left(e^{i \omega_0 t} - e^{- i \omega_0 t}\right)^{\flatfrac12}} \\
		iG(0, t, 0, 0) &= \sqrt{\frac{m \omega_0}{\pi}} \left(e^{i \omega_0 t} - e^{- i \omega_0 t}\right)^{-\flatfrac12} \\
		iG(0, t, 0, 0) &= \sqrt{\frac{m \omega_0}{\pi}} e^{-i\frac{\omega_0}{2} t}\left(1 - e^{- 2 i \omega_0 t}\right)^{-\flatfrac12} \label{eq:donebesidesseries}
	\end{align}

	We can now take a brief aside to investigate the series expansion of $(1 - x)^{-\flatfrac12}$.
	\triv the nested derivatives are
	\begin{align}
		\dv{x} (1 - x)^{-\flatfrac12} &= \frac{1}{2} (1 - x)^{-\flatfrac32}\\
		\dv[2]{x} (1 - x)^{-\flatfrac12} &= \frac{1}{2} \cdot \frac{3}{2}(1 - x)^{-\flatfrac52}\\
		\dv[3]{x} (1 - x)^{-\flatfrac12} &= \frac{1}{2} \cdot \frac{3}{2} \cdot \frac{5}{2} (1 - x)^{-\flatfrac72}\\
		\dv[n]{x} (1 - x)^{-\flatfrac12} &= \frac{(2n - 1){!}{!}}{2^n} (1 - x)^{-\flatfrac{2n + 1}{2}}.
	\end{align}
	\thrf we can write the series as
	\begin{align}
		(1 - x)^{-\flatfrac12} &= 1 + \sum_{n = 1}^\infty \frac{1}{n{!}} \frac{(2n - 1){!}{!}}{2^n} x^n,
	\end{align}
	where we've worked out the constant term through careful application of inserting $x = 0$ into the series.
	Note that we get convergence issues when $x = 1$.
	This is only a mild annoyance, however.
	\triv $(2n - 1){!}{!} = \frac{(2n - 1){!}}{2^{n - 1}(n - 1){!}}$.
	\thrf
	\begin{align}
		(1 - x)^{-\flatfrac12} &= 1 + \sum_{n = 1}^\infty \frac{1}{n{!}} \frac{(2n - 1){!}}{2^{2 n - 1}(n - 1){!}} x^n \\
		(1 - e^{- 2 i \omega_0 t})^{-\flatfrac12} &= 1 + \sum_{n = 1}^\infty \frac{1}{n{!}} \frac{(2n - 1){!}}{2^{2 n - 1}(n - 1){!}} \left(- 2 i \omega_0 \right)^n e^{- i 2 n \omega_0 t} \\
		(1 - e^{- 2 i \omega_0 t})^{-\flatfrac12} &= 1 + \sum_{n = 1}^\infty \frac{ \left(- i \omega_0 \right)^n }{n{!}} \frac{(2n - 1){!}}{2^{n - 1}(n - 1){!}}e^{- i 2 n \omega_0 t}
	\end{align}
	If we now set $x = e^{- 2 i \omega_0 t}$, we get a series we can use in~\eqref{eq:donebesidesseries}, which we can relate to the form given in the hint (that aside about the series just fixes $C_n$).
	We get then that $\epsilon_n = \frac{\omega_0}{2} + 2 n \omega_0$, (which again, we could have gotten from~\eqref{eq:donebesidesseries}).
	We're missing some of the energy eigenvalues for the harmonic oscillator because we're only looking at $x_b = x_a = 0$, at which point the propagator ignores all of the odd wavefunctions.
	Including those by choosing a different $x$ or even by averaging over all $x$ would be easy enough to find the pole structure, as the exponential that disappears in~\eqref{eq:harmonicoscprop} would just be some function of $e^{-i\omega_0 t}$.
	However, finding the $C_n$ in the series would be much more difficult.
	This problem is obviously trivial if we just use the known energy eigenvalues and eigenstates of the harmonic oscillator, as all of the odd-indexed Hermite polynomials are in fact odd functions, making those terms in the sum~\eqref{eq:target1} disappear.

	\newpage
	\listoftodos

\end{document}
