\documentclass{article}

% set up telugu
\usepackage{fontspec}
\newfontfamily\telugufont{Potti Sreeramulu}[Script = Telugu,RawFeature={mode=harf}]
\usepackage{polyglossia}
\setdefaultlanguage{english}
\setotherlanguage{telugu}

%other packages
\usepackage{amsmath}
\usepackage{amssymb}
\usepackage{physics}
\usepackage{siunitx}
\usepackage[plain]{fancyref}
\usepackage{luacode}

% custom deepak packages
\usepackage{trivially}
\usepackage{subtitling}

\begin{luacode*}
	math.randomseed(31415926)
\end{luacode*}

\title{Problem 2.1.1}
\subtitle{On the pole structure of propagators}
\author{\begin{telugu}హృదయ్ దీపక్ మల్లుభొట్ల\end{telugu}}

% !TeX spellcheck = en_GB
\begin{document}
	\maketitle
	Show that, in the frequency space, we have
	\begin{equation}
		G(x_b, x_a, \omega) = \sum_n \frac{\psi_n(x_b) \psi^\dagger(x_a)}{\omega - \epsilon_n} \label{eq:target1}
	\end{equation}
	where $\psi_n$ are the energy eigenfunctions.
	The propagator for a harmonic oscillator has the form
	\begin{equation}
		G(x_b, t, x_a, 0) = (-i) \left( \frac{m \omega_0}{2 \pi i \sin(t \omega_0)} \right)^{\flatfrac12} \exp(\frac{i m \omega_0}{2 \pi \sin(t\omega_0)}\left[(x_b^2 + x_a^2)\cos(t \omega_0) - 2 x_b x_a \right]) \label{eq:harmonicoscprop}
	\end{equation}
	Study and explain the pole structure of $G(0, 0, \omega)$ for the harmonic oscillator.
	(Hint: Try to expand $G(0, t, 0, 0,)$ in the form $\sum C_n e^{-i t \epsilon_n}$.)

	\section{Solution} \label{sec:solution}
	\subsection{Pole structure form of propagator} \label{subsec:sol1}
	We want to begin by showing~\eqref{eq:target1}.
	\triv the coordinate Green function is defined via
	\begin{equation}
		i G(x_b, t_b, x_a, t_a) = \bra{x_b} U(t_b, t_a) \ket{x_a}.
	\end{equation}
	\triv
	\begin{equation}
		\bra{n_b} U(t_b, t_a) \ket{n_a} = e^{-i \epsilon_{n_b} (t_b - t_a)} \delta_{n_a, n_b},
	\end{equation}
	where $\ket{n_i}$ is the $i-$th energy eigenstate.
	\thrf
	\begin{align}
		G(x_b, t_b, x_a, t_a) &= -i \sum_{n} \psi_n(x_b) \psi_n^\dagger(x_a)  e^{-i \epsilon_{n} (t_b - t_a)}
	\end{align}
	In frequency space,
	\begin{align}
		G(x_b, x_a, \omega) = \int_0^\infty \dd{t} G(x_b, t_a + t, x_a, t_a) e^{i t \omega - \delta t},
	\end{align}
	where $\delta$ is some small positive constant we'll send to zero later.
	\triv when we find this in frequency space
	\begin{align}
		G(x_b, x_a, \omega) &= \sum_{n} \frac{\psi_n(x_b) \psi_n^\dagger(x_a)}{\omega - \epsilon_n + i \delta}
	\end{align}
	\thrf after we send $\delta \rightarrow 0$, we have shown~\eqref{eq:target1}, as desired.

	\subsection{Pole structure of harmonic oscillator propagator} \label{subsec:sol2}
	We can start with~\eqref{eq:harmonicoscprop}, and set $x_b = x_a = 0$.
	\triv this gives us
	\begin{align}
		i G(x_b = 0, t, x_a = 0, 0) &= \sqrt{\frac{m \omega_0}{\pi}} e^{-i\frac{\omega_0}{2} t}\left(1 - e^{- 2 i \omega_0 t}\right)^{-\flatfrac12}. \label{eq:donebesidesseries}
	\end{align}
	Note here the $e^{-i\frac{\omega_0}{2} t}$, which will add $\frac{\omega_0}{2}$ to the energy.
	\triv some simple series expansion gives us
	\begin{align}
		(1 - e^{- 2 i \omega_0 t})^{-\flatfrac12} &= 1 + \sum_{n = 1}^\infty \frac{ \left(- i \omega_0 \right)^n }{n{!}} \frac{(2n - 1){!}}{2^{n - 1}(n - 1){!}}e^{- i 2 n \omega_0 t}
	\end{align}
	\thrf we get that $\epsilon_n = \frac{\omega_0}{2} + 2 n \omega_0$, (which we could have gotten from~\eqref{eq:donebesidesseries}).
	We're missing some of the energy eigenvalues for the harmonic oscillator because we're only looking at $x_b = x_a = 0$, at which point the propagator ignores all of the odd wavefunctions.
	Including those by choosing a different $x$ or even by averaging over all $x$ would be easy enough to find the pole structure, as the exponential that disappears in~\eqref{eq:harmonicoscprop} would just be some function of $e^{-i\omega_0 t}$.
	However, finding the $C_n$ in the series would be much more difficult.
	This problem is obviously trivial if we just use the known energy eigenvalues and eigenstates of the harmonic oscillator, as all of the odd-indexed Hermite polynomials are in fact odd functions, making those terms in the sum~\eqref{eq:target1} disappear.

\end{document}
