\documentclass{article}

% set up telugu
\usepackage{fontspec}
\newfontfamily\telugufont{Potti Sreeramulu}[Script = Telugu,RawFeature={mode=harf}]
\usepackage{polyglossia}
\setdefaultlanguage{english}
\setotherlanguage{telugu}

%other packages
\usepackage{amsmath}
\usepackage{amssymb}
\usepackage{physics}
\usepackage{siunitx}
\usepackage{todonotes}
\usepackage[plain]{fancyref}
\usepackage{trivially}
\usepackage{luacode}

\begin{luacode*}
	math.randomseed(314159)
\end{luacode*}

\title{Problem 2.1.1}
\author{\begin{telugu}హృదయ్ దీపక్ మల్లుభొట్ల\end{telugu}}

% !TeX spellcheck = en_GB
\begin{document}
	\maketitle
	Show that, in the frequency space, we have
	\begin{equation}
		G(x_b, x_a, \omega) = \sum_n \frac{\psi_n(x_b) \psi^\dagger(x_a)}{\omega - \epsilon_n}
	\end{equation}
	where $\psi_n$ are the energy eigenfunctions.
	The propagator for a harmonic oscillator has the form
	\begin{equation}
		G(x_b, t, x_a, 0) = (-i) \left( \frac{m \omega_0}{2 \pi i \sin(t \omega_0)} \right)^{\flatfrac12} \exp(\frac{i m \omega_0}{2 \pi \sin(t\omega_0)}\left[(x_b^2 + x_a^2)\cos(t \omega_0) - 2 x_b x_a \right])
	\end{equation}
	Study and explain the pole structure of $G(0, 0, \omega)$ for the harmonic oscillator.
	(Hint: Try to expand $G(0, t, 0, 0,)$ in the form $\sum C_n e^{-i t \epsilon_n}$.)
	\newpage
	\listoftodos

\end{document}
